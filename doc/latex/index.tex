The files in this program provide a heirarchical data structure system for carrying out dislocation dynamics simulations in two dimensions. The base class is \hyperlink{classDefect}{Defect}, which represents a generic defect in a metallic crystal. All other defects, such as dislocations, dislocation sources, precipitates, etc., are represented by their own classes which inherit certain functions from the \hyperlink{classDefect}{Defect} class.

The goal of carrying out these simulations in two dimensions is to be able to simulate plastic deformation of up to a few percent. Current three dimensional dislocation dynamics simulations are computationally expensive. This approach hopes to sacrifice some of the precision in order to gain in speed and flexibility.

The program is under development now, with the data structures being defined. When it will be complete, it is intended to have data structures nested within each other, hence the name \href{https://en.wikipedia.org/wiki/Matryoshka_doll}{\tt Matryoshka}. For example, a polycrystal is a collection of grains; a grain is a collection of slip systems; a slip system is a collection of slip planes; a slip plane is a collection of dislocations, dislocation sources and other defects. This program will also take advantage of the functionality provided by the C++ S\-T\-L to manage lists of various objects in the simulation. Once the base simulations execute successfully, other defects will be introduced.

To view the hierarchical structure, go to the section labeled Data Structures $>$ Class Hierarchy. A good place to start would be the \hyperlink{classDefect}{Defect} class, which is the generic base class for most of the entities present in the simulation. 